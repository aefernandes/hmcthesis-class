%%% Template File for Use with the hmcthesis.cls.
%%%
%%% C.M. Connelly <cmc@math.hmc.edu>
%%%  $Id$
%%%
%%%  Tag: $Name$


%%% !!! HMC STUDENTS SHOULD REMOVE THE FOLLOWING COPYRIGHT NOTICE FROM
%%% !!! FINAL SUBMISSIONS.

%%% Copyright (C) 2004-2015 Department of Mathematics, Harvey Mudd College.
%%%
%%% This file is part of the sample thesis provided to HMC
%%% mathematics students and is based on the sample thesis's
%%% master.tex file.  Please see that file for more information.

%%% See the COPYING document, which should accompany this
%%% distribution, for information about distribution and modification
%%% of the document and its components.

%%% !!! END COPYRIGHT NOTICE.

%%% Preamble.

%%% The top part of your document is called the preamble.  You supply
%%% some basic information about the document (such as its title and
%%% author) in a form that LaTeX can understand here.

%%% You can also load additional LaTeX packages, or style files, that
%%% affect the way that the document is laid out, typeset, or supply
%%% additional commands or environments.


%%% Theses use the hmcthesis document class, which should be
%%% located somewhere in TeX's search path.
\documentclass[<<department>>]{hmcthesis}

%%% You must set a document-class option to tell the class what
%%% department your report will be for!  Replace the string
%%% <<document>> with the name of a department, where supported
%%% departments are
%%%
%%%    biology
%%%    computer-science
%%%    chemistry
%%%    engineering
%%%    mathematics
%%%    physics


%%% At the end of the year, add the sigpage document-class
%%% option to generate a signature page for your thesis, as in

%%% \documentclass[sigpage,<<department>>]{hmcthesis}

%%% Note that you still have to provide the department name.  And
%%% for your final submission, you should only have the department
%%% document-class option.


%%% The preamble can also be used to define your own commands and
%%% environments, set some constants that will be used throughout your
%%% document, and so on.


%%% Other packages needed by your document may be loaded here.
%%% with the \usepackage{} command.  If you choose to load
%%% additional packages, make sure that they appear *before* the
%%% line loading hyperref; hyperref changes pieces of other
%%% packages, so it's important that it be loaded last.

% \usepackage{graphicx}           % More control over graphic inclusion.
% \usepackage{amsthm}             % AMS theorem styles

%%% Load all other packages before this point.

%%% Load hyperref.
\usepackage[breaklinks=true,
  bookmarks,
  pdfpagemode=UseOutlines,
  pdfpagelayout=SinglePage]{hyperref}


%%% As you may have guessed, LaTeX's comment character is the percent
%%% sign.  Any line that starts with a % will be ignored.  You can
%%% also use the comment character to add comments to the end of a
%%% line that will be parsed by TeX.

%%% The optional \includeonly command allows you to specify the names
%%% of chapters that you want to typeset.  It is useful for debugging
%%% or for working intensely on one particular part of your document
%%% when you don't want to take the time to retypeset the entire document.


%%% The first active line in your LaTeX document is the \documentclass
%%% command, which loads a LaTeX class file.  Class files generally
%%% define the appearance of a document, and include a variety of
%%% structural commands.

%%% This optional command provides additional context around an error.
%%% It can be helpful when tracking down a problem. 
%\setcounter{errorcontextlines}{1000}


%%% Information about this document.

%%% I find it most useful to put identifying information about a
%%% document near the top of the preamble.  Technically, this
%%% information must precede the \maketitle command, which often
%%% appears immediately after the beginning of the document 
%%% environment.  Placing it near the top of the document makes it
%%% easier to identify the document, and keeps it from getting
%%% mixed up with the content of your document.

%%% So, some questions.

%% What is the title of your report?
\title{Title of Your Thesis}

%% Who are the authors of the report?  (Separate multiple names with \and.)
\author{Your Name}

%% What is your faculty advisor's name?  (Again, separate names with
%% \and, if necessary.)
%% 
%% Don't include titles (e.g., ``Prof.'', ``Dr.'').
\advisor{Advisor One \and Advisor Two}

%% Second reader's name?  If you have a third or fourth reader, add
%%% their names here, separated by the \and command.
%% 
%% Don't include titles (e.g., ``Prof.'', ``Dr.'').
\reader{Second Reader}

%% The year in which you are submitting your thesis.
\thesisyear{2016} 

%% Optional: The month in which you are submitting your thesis (if
%% it's a month other than May).  For example,
% \thesismonth{December} 

%%% End of information section.


%%% New commands and environments.

%%% You can define your own commands and environments here.  If you
%%% have a lot of material here, you might want to consider splitting
%%% the commands and environments into a separate ``style'' file that
%%% you load with \usepackage.

% \newcommand{\coolcommand}[1]{#1 is cool.} % Lets everyone know that
                                % the person or thing that you provide
                                % as the argument to the command is
                                % cool.

% \newcounter{cms}


%%% Some theorem-like command definitions.

%%% The \newtheorem command comes from the amsthm package.  That
%%% package is *not* loaded by the class file, so if you choose
%%% to use these commands, you'll need to load the package above.

% \newtheorem{thm}{Theorem}[chapter]
% \newtheorem{lem}{Lemma}[chapter]


%%% If you find that some words in your document are being hyphenated
%%% incorrectly, you can specify the correct hyphenation using the
%%% \hyphenation command.  Note that words are separated by
%%% whitespace, as shown below.

\hyphenation{ap-pen-dix wer-ther-i-an}


%%% The start of the document!

%% The document environment is the main environment in any LaTeX
%% document.  It contains other environments, as well as your text.

\begin{document}

%%% The front matter of a large document includes the title page or
%%% pages, tables of contents, lists of figures or tables, and so on,
%%% your abstract, a preface or introduction, and so on.  It's
%%% delineated with the \frontmatter command.

\frontmatter


%%% One of the things that the \frontmatter does is make page
%%% numbers appear as lowercase Roman numerals---i, vi, xii, and so
%%% on.

%%% The first thing in the front matter is your title page.  The title
%%% page is formatted by commands in the document class file, so you
%%% don't need to worry about what it looks like -- just putting the
%%% \maketitle command in your document (and filling in the necessary
%%% information for the identification commands above) is enough.

\maketitle


%%% Abstract

\begin{abstract}
  Your abstract should be a \emph{brief} summary of the contents of
  your report.  Don't go into excruciating detail here---there's
  plenty of room for that later.
\end{abstract}


%%% Table of Contents, List of Figures, and List of Tables.
%%% 
%%% If you don't have any figures or tables in your report, you
%%% should comment out the appropriate command.  If you don't,
%%% you'll get an extra, mostly blank, page in your typeset report.

\tableofcontents
\listoffigures
\listoftables



%%% Acknowledgments.

\begin{acknowledgments}
%% Thank some people here, if you like.
\end{acknowledgments}

%%% End of the front matter.


%%% Beginnning of the main matter.

%% The main part of your report consists of normal, numbered
%% chapters as well as any appendices.  Bibliographies, indexes, and
%% so on are part of the back matter.  The main matter is opened with the
%% \mainmatter command.

\mainmatter


%%% Content.

%%% For smaller documents---especially those you're writing by
%%% yourself---you might write your entire report using a single LaTeX
%%% source file.  For larger documents, we recommend that you split
%%% the source file into several separate, smaller files.  The smaller
%%% files are ``included'' into your main, or ``master'' document
%%% using \include commands.

%%% Splitting your source has several advantages.  First, if you're
%%% working on a document with a group of people, it allows you to
%%% have more than one person working on different parts of the
%%% document at the same time (although we still recommend that you
%%% use CVS or a similar revision-control system!).  Second, smaller
%%% document chunks allow you to reorganize your document more
%%% easily.  If you later decide that Chapter 8 would be better as
%%% Chapter 4, all you have to do is swap the \include commands
%%% around.  For that reason, you should give your separate chapters
%%% meaningful names, such as ``introduction'', ``background'', or
%%% ``conclusions'' rather than calling them ``chapter1'',
%%% ``chapter2'', and so on.
%%% 
%%% Finally, splitting the document allows you to concentrate on a
%%% particular section without being distracted by other
%%% sections---all you have to do is comment out the \include line for
%%% the sections you're not working on.  This technique can be
%%% especially useful when you're trying to track down a problem by
%%% allowing you to easily locate the file with the problem by
%%% ruling out the other sections.

%%% In our example document, we define several chapters that have
%%% useful information about writing Clinic or thesis reports or
%%% using LaTeX.  Here, we'll just use placeholders (but not
%%% chapter1, chapter2, etc!).  .


%%% Chapter 1

\include{a_chapter}

%%% Chapter 2

\include{another_chapter}

%%% Chapter 3

\include{still_another_chapter}

%%% Chapter 4

\include{yet_another_chapter}

%%% Chapter 5

\include{one_more_chapter}

%%% Chapter 6

\include{possibly_the_last_chapter}


%%% Appendices.

%%% Appendices are just like chapters, only they're generally
%%% lettered rather than numbered (although that depends on your
%%% document class, of course).

%%% The appendices are delineated with the \appendix command.
%%% Individual appendices are begun with the standard \chapter or
%%% \section commands.  In our example, we'll \include them just as
%%% we did other chapters.

%%% If you don't have any appendices, comment out the \appendix
%%% command.

\appendix

\include{our_appendix}

\include{our_source_code}


%%% Back matter.

%%% The back matter of a document is where the bibliography, index,
%%% glossary, and other unnumbered chapters or sections occur.  It
%%% starts, not surprisingly, with the \backmatter command.

\backmatter


%%% Bibliography.

%%% BibTeX is the tool to use for citations and layout of your
%%% bibliography.  Instead of having to type ``[5]'' or ``(Jones,
%%% 1968)'' (and keep track of which citation is which and renumber
%%% them as you add more references to your bibilography), you use
%%% special commands that allow BibTeX and LaTeX to automatically put
%%% the correct information in the right place.

%%% Depending on your field, it may or may not be appropriate to list
%%% references for which you haven't included specific citations.  If
%%% your field sanctions such practices, or if you just want to get an
%%% idea of what you have in your bibliography file, you can include
%%% everything with the \nocite{*} command.
\nocite{*} 


%%% The appearance of your bibliography and citations in your text are
%%% defined by a combination of any bibliography-related LaTeX
%%% packages (such as natbib, harvard, or chicago) and the particular
%%% bibliography style file that you load with the \bibliographystyle
%%% command.  Bibliography-style files end in .bst; you can find them
%%% by searching your file system using whatever tools you have for
%%% doing searches.  (On most modern Unices, ``locate .bst'' will give
%%% you an idea of what's available.)

%%% For math Clinic reports and theses, we use the hmcmath
%%% bibliography style.

\bibliographystyle{hmcmath}

%%% The hmcmath bibliographic style is meant for use with the
%%% natbib bibliography and citation package, which is loaded by
%%% the class file.

%%% The particular bibliography data file or files that you want to
%%% use are specified with the \bibliography file.  Multiple files are
%%% separated by commas.

%%% You might want to use multiple bibliography (or ``bib'') files if
%%% you had a master bib file containing references you use again and
%%% again, and another containing only records for references for a
%%% particular project.

%%% Many people create a single, large bib file that they use for
%%% everything they write.  That approach requires you to \cite every
%%% reference that you want to use in your document -- using
%%% \nocite{*} with a huge bibliography database will give you a large
%%% bibliography containing many references you haven't consulted for
%%% your particular document!

\bibliography{our_bib_file}


%%% Glossary or Index.
%%% If you were going to include a glossary or index in your document,
%%% the relevant commands would appear here.

%%% If you think that you would like to include such features, talk
%%% with someone who's worked with LaTeX a lot very early in your
%%% writing process.  These commands require you to do a bit of
%%% thinking about what you would want to index or gloss in
%%% advance---going back though a completed document to add \index
%%% commands is *not fun*.


\end{document}
